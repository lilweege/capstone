\documentclass[12pt, titlepage]{article}

\usepackage{amsmath, mathtools}

\usepackage[round]{natbib}
\usepackage{amsfonts}
\usepackage{amssymb}
\usepackage{graphicx}
\usepackage{colortbl}
\usepackage{xr}
\usepackage{hyperref}
\usepackage{longtable}
\usepackage{xfrac}
\usepackage{tabularx}
\usepackage{float}
\usepackage{siunitx}
\usepackage{booktabs}
\usepackage{multirow}
\usepackage[section]{placeins}
\usepackage{caption}
\usepackage{fullpage}

\hypersetup{
bookmarks=true,     % show bookmarks bar?
colorlinks=true,       % false: boxed links; true: colored links
linkcolor=red,          % color of internal links (change box color with linkbordercolor)
citecolor=blue,      % color of links to bibliography
filecolor=magenta,  % color of file links
urlcolor=cyan          % color of external links
}

\usepackage{array}

\externaldocument{../../SRS/SRS}

\input{../../Comments}
\input{../../Common}

\begin{document}

\title{Module Interface Specification for \progname{}}

\author{\authname}

\date{\today}

\maketitle

\pagenumbering{roman}

\section{Revision History}

\begin{tabularx}{\textwidth}{p{3cm}p{2cm}X}
\toprule {\bf Date} & {\bf Version} & {\bf Notes}\\
\midrule
January 17 & 1.0 & Initial documentation\\
\bottomrule
\end{tabularx}

~\newpage

\section{Symbols, Abbreviations and Acronyms}

See SRS Documentation at \href{https://github.com/SyntaxSentinels/SyntaxSentinels/blob/main/docs/SRS-Volere/SRS.pdf}{SRS}

% \wss{Also add any additional symbols, abbreviations or acronyms}

\newpage

\tableofcontents

\newpage

\pagenumbering{arabic}

\section{Introduction}

The following document details the Module Interface Specifications for SyntaxSentinals.

This project seeks to create a plagiarism algorithm that relies on NLP
techniques of present to account for semantics and prevent primitive cir-
cumvention of plagiarism detection, such as the addition of benign lines or
variable name changes. The users of our product will primarily be those con-
cerned with fairness and integrity of code submissions within a competitive
environment, such as professors or code competition holders.

Users are intended to use the resulting product of our project by giving
it code snippets and receiving a plagiarism report in return. This report
will contain a set of similarity scores for inputted code snippets, which when
assessed against an outputted threshold will indicate likelihood of plagiarsm
having taken place. This will benefit the users by allowing them to more accurately assess the presence of plagiarized work, 
creating a fairer environment for competition and rewarding coders correctly. 
Ultimately, the project aims to help users achieve an environment that cycles merit 
instead of cheating, which is believed to be a primary interest of users too.

Complementary documents include the System Requirement Specifications
and Module Guide.  The full documentation and implementation can be
found at \url{https://github.com/SyntaxSentinels/SyntaxSentinels}. 

\section{Notation}

Below is a summary of the notations used in this document:

\begin{center}
\renewcommand{\arraystretch}{1.2}
\noindent
\begin{tabular}{l l p{8cm}} 
\toprule 
\textbf{Data Type} & \textbf{Notation} & \textbf{Description} \\ 
\midrule
character & char & A single symbol or digit. \\ 
integer & $\mathbb{Z}$ & A whole number in the range $(-\infty, \infty)$. \\ 
natural number & $\mathbb{N}$ & A whole number in the range $[1, \infty)$. \\ 
real & $\mathbb{R}$ & Any number in the range $(-\infty, \infty)$. \\ 
boolean & bool & A logical value that can either be \texttt{true} or \texttt{false}. \\
string & str & A sequence of characters. \\ 
tuple & tuple & An ordered collection of elements, potentially of different types. \\
Abstract Syntax Tree & AST & Tree representing code like that described here \url{https://docs.python.org/3/library/ast.html} \\
file & file & An opaque handle to a file on the disk, or a collection of binary data to be written to a file \\
\bottomrule
\end{tabular} 
\end{center}

\noindent
The following conventions are also used:
\begin{itemize}
    \item \textbf{Assignment}: The operator \texttt{:=} denotes assignment.
    \item \textbf{Conditional Rules}: Conditional statements follow the structure $(c_1 \Rightarrow r_1 \mid c_2 \Rightarrow r_2 \mid \ldots \mid c_n \Rightarrow r_n)$, where $c_i$ are conditions and $r_i$ are corresponding results.
    \item \textbf{Access Programs}: Functions and methods are defined with their inputs, outputs, and exceptions as described in the syntax sections of each module.
\end{itemize}

\section{Module Decomposition}

The following table is taken directly from the Module Guide document for this project.

\begin{table}[h!]
    \centering
    \begin{tabular}{p{0.35\textwidth} p{0.55\textwidth}}
    \toprule
    \textbf{Level 1} & \textbf{Level 2} \\
    \midrule
    {Hardware-Hiding Module} & ~ \\
    \midrule
    \multirow{2}{*}{Behaviour-Hiding Module} 
    & User Authentication Module \\
    & Code Upload Module \\
    & Results Upload Module \\
    & Report Results Module \\
    & Email Sending Module \\
    & Flagging Module \\
    & Threshold Adjustment Module \\
    \midrule
    \multirow{2}{*}{Software Decision Module} 
    & Report Generation Module \\
    & Similarity Scoring Module \\
    & NLP Model Module \\
    & Abstract ML Model Module \\
    & Tokenization Module \\
    & AST Module \\
   
    \bottomrule
    \end{tabular}
    \caption{Module Hierarchy}
    \label{TblMH}
  \end{table}
%new breakdown
\newpage
~\newpage

% \section{MIS of \wss{Module Name}} \label{Module} \wss{Use labels for
%   cross-referencing}

% \wss{You can reference SRS labels, such as R\ref{R_Inputs}.}

% \wss{It is also possible to use \LaTeX for hypperlinks to external documents.}

% \subsection{Module}

% \wss{Short name for the module}

% \subsection{Uses}


% \subsection{Syntax}

% \subsubsection{Exported Constants}

% \subsubsection{Exported Access Programs}

% \begin{center}
% \begin{tabular}{p{2cm} p{4cm} p{4cm} p{2cm}}
% \hline
% \textbf{Name} & \textbf{In} & \textbf{Out} & \textbf{Exceptions} \\
% \hline
% \wss{accessProg} & - & - & - \\
% \hline
% \end{tabular}
% \end{center}

% \subsection{Semantics}

% \subsubsection{State Variables}

% \wss{Not all modules will have state variables.  State variables give the module
%   a memory.}

% \subsubsection{Environment Variables}

% \wss{This section is not necessary for all modules.  Its purpose is to capture
%   when the module has external interaction with the environment, such as for a
%   device driver, screen interface, keyboard, file, etc.}

% \subsubsection{Assumptions}

% \wss{Try to minimize assumptions and anticipate programmer errors via
%   exceptions, but for practical purposes assumptions are sometimes appropriate.}

% \subsubsection{Access Routine Semantics}

% \noindent \wss{accessProg}():
% \begin{itemize}
% \item transition: \wss{if appropriate} 
% \item output: \wss{if appropriate} 
% \item exception: \wss{if appropriate} 
% \end{itemize}

% \wss{A module without environment variables or state variables is unlikely to
%   have a state transition.  In this case a state transition can only occur if
%   the module is changing the state of another module.}

% \wss{Modules rarely have both a transition and an output.  In most cases you
%   will have one or the other.}

% \subsubsection{Local Functions}

% \wss{As appropriate} \wss{These functions are for the purpose of specification.
%   They are not necessarily something that is going to be implemented
%   explicitly.  Even if they are implemented, they are not exported; they only
%   have local scope.}


\section{MIS of Code Upload Module} \label{mCodeUpload}

\subsection{Module} %MODULE STARTS###

\texttt{CodeUploadModule} 

\subsection{Uses}

\begin{itemize}
    \item File system or equivalent I/O library (for reading and writing local files)
    \item Parser or utility library for code/data formatting, if necessary
\end{itemize}

\subsection{Syntax}

\subsubsection{Exported Constants}

\begin{itemize}
    \item \texttt{MAX\_FILE\_LENGTH}: $\mathbb{N}$ \\ 
    The maximum allowed code lines in a single file for upload.
    \item \texttt{ALLOWED\_FILE\_TYPES}: $\mathbb{N}$ \\ 
    A list of permissible file extensions (e.g., \texttt{.py, .txt, .zip}).
\end{itemize}

\subsubsection{Exported Access Programs}

\begin{center}
\begin{tabular}{p{5cm} p{3.5cm} p{3.5cm} p{2cm}}
\hline
\textbf{Name} & \textbf{In} & \textbf{Out} & \textbf{Exceptions} \\
\hline
\texttt{uploadFile} & filePath : Str & success: bool & FileError \\
\texttt{validateFileFormat} & filePath : Str & valid: bool & FormatError \\
\texttt{convertFileToData} & filePath : Str & - & ConversionError \\
\texttt{getParsedData} & - & snippets: JSON & -\\
\hline
\end{tabular}
\end{center}

\subsection{Semantics}

\subsubsection{State Variables}


\begin{itemize}
    \item \texttt{uploadedFile}: Str \\
     Stores the path (or reference) to the currently uploaded file.
    \item \texttt{parsedData}: Str \\ 
    Stores the in-memory data structure resulting from converting the file.
\end{itemize}

\subsubsection{Environment Variables}

\begin{itemize}
    \item \texttt{TEMP\_UPLOAD\_PATH}: Str \\
     Directory path for temporarily storing uploaded files.
\end{itemize}

\subsubsection{Assumptions}

\begin{itemize}
    \item The file path provided exists and points to a valid file.
    \item Sufficient storage space is available in \texttt{TEMP\_UPLOAD\_PATH}.
    \item Uploaded files comply with any project-specific format or version constraints.
\end{itemize}

\subsubsection{Access Routine Semantics}

\noindent \texttt{uploadFile}(filePath: Str):
\begin{itemize}
    \item \textbf{transition:}
    \begin{itemize}
        \item Copy the file from \textit{filePath} to \texttt{TEMP\_UPLOAD\_PATH}.
        \item Update \texttt{uploadedFile} to reflect the new file location.
    \end{itemize}
    \item \textbf{input:} Path representing location of snippets on user computer.
    \item \textbf{output:} Returns \texttt{true} on success.
    \item \textbf{exception:} \texttt{FileError} if file I/O fails or \textit{filePath} is invalid.
\end{itemize}

\noindent \texttt{validateFileFormat}(filePath: Str):
\begin{itemize}
    \item \textbf{transition:} None (no internal state change).
    \item \textbf{input:} Path representing location of snippets on user computer.
    \item \textbf{output:} Returns \texttt{true} if the file meets \texttt{MAX\_FILE\_SIZE} and 
    \texttt{ALLOWED\_FILE\_TYPES} conditions.
    \item \textbf{exception:} \texttt{FormatError} if the file type or size is invalid.
\end{itemize}

\noindent \texttt{convertFileToData}(filePath: Str):
\begin{itemize}
    \item \textbf{transition:}
    \begin{itemize}
        \item Reads raw file content from the \texttt{uploadedFile}.
        \item Parses and converts the content into JSON for assignment to \texttt{parsedData}.
    \end{itemize}
    \item \textbf{input:} Path representing location of snippets on user computer.
    \item \textbf{output:} None
    \item \textbf{exception:} \texttt{ConversionError} if file parsing fails or content is malformed.
\end{itemize}

\noindent \texttt{getParsedData}():
\begin{itemize}
    \item \textbf{transition:} None (no internal state change).
    \item \textbf{input:} None
    \item \textbf{output:} JSON representing files uploaded by user.
    \item \textbf{exception:} None
\end{itemize}

\subsubsection{Local Functions}

\noindent \texttt{readLocalFile(path: Str)}: 
\begin{itemize}
    \item \textbf{transition:} None (no internal state change.)
    \item \textbf{input:} Path representing location of snippets on user computer.
    \item \textbf{output:} \textit{list [String]} object holding all info from each
    file.
    \item \textbf{exception:} \texttt{FileError} if file I/O fails or \textit{filePath} is invalid.
\end{itemize}

\noindent \texttt{parseCodeData(rawContent: list[Str])}: Transforms raw file content into a \texttt{JSON}.
\begin{itemize}
    \item \textbf{transition:} None (no internal state change).
    \item \textbf{input:}  \textit{list [String]} object holding all info from each
    file.
    \item \textbf{output:} JSON representing code snippet contents.
    \item \textbf{exception:} None
\end{itemize}

\section{MIS of Results Upload Module} \label{mResultsUpload}

\subsection{Module} %MODULE STARTS ###

\texttt{ResultsUploadModule}

\subsection{Uses}

\begin{itemize}
    \item File system or equivalent I/O utilities (to read and load local report files, if applicable)
    \item Front-end/UI framework (to display the parsed results)
    \item HTTP or backend connector (if the parsed results need to be sent elsewhere)
\end{itemize}

\subsection{Syntax}

\subsubsection{Exported Constants}

\begin{itemize}
    \item \texttt{MAX\_REPORT\_FILE\_SIZE}: $\mathbb{N}$ \\
     Maximum allowed file size (in bytes) for a report file.
    \item \texttt{ALLOWED\_REPORT\_TYPES}: $\mathbb{N}$ \\
     Only .zip is allowed.
\end{itemize}

\subsubsection{Exported Access Programs}

\begin{center}
\begin{tabular}{p{5cm} p{3.5cm} p{3.5cm} p{2cm}}
\hline
\textbf{Name} & \textbf{In} & \textbf{Out} & \textbf{Exceptions} \\
\hline
\texttt{uploadResultsFile} & filePath: Str & success: bool & FileError \\
\texttt{parseResultsFile} & filePath: Str & report: JSON & ParseError \\
\hline
\end{tabular}
\end{center}

\subsection{Semantics}

\subsubsection{State Variables}

\begin{itemize}
    \item \texttt{uploadedReportFile}: Str \\
    Stores the path (or reference) to the currently uploaded report file.
    \item \texttt{parsedReportData}: JSON \\
     Stores the in-memory data structure resulting from parsing the report file.
\end{itemize}

\subsubsection{Environment Variables}

\begin{itemize}
    \item None
\end{itemize}

\subsubsection{Assumptions}

\begin{itemize}
    \item The file path provided points to a valid file and does not exceed \texttt{MAX\_REPORT\_FILE\_SIZE}.
    \item The file type is one of \texttt{ALLOWED\_REPORT\_TYPES}.
    \item The front-end/UI framework is loaded and available for rendering the report data.
\end{itemize}

\subsubsection{Access Routine Semantics}

\noindent \texttt{uploadResultsFile}(filePath: Str):
\begin{itemize}
    \item \textbf{transition:}
    \begin{itemize}
        \item Copies file from \textit{filePath} to \texttt{TEMP\_REPORT\_PATH} (if needed).
        \item Updates \texttt{uploadedReportFile} to reflect the new file location.
    \end{itemize}
    \item \textbf{input:} Path representing location of snippets on user computer.
    \item \textbf{output:} Returns \texttt{true} if upload is successful.
    \item \textbf{exception:} \texttt{FileError} if reading or copying the file fails.
\end{itemize}

\noindent \texttt{parseResultsFile}(filePath: Str):
\begin{itemize}
    \item \textbf{transition:}
    \begin{itemize}
        \item Opens and reads the specified report file.
        \item Updates \texttt{parsedReportData} with \texttt{JSON} assembled from the file content.
    \end{itemize}
    \item \textbf{input:} Path representing location of snippets on user computer.
    \item \textbf{output:} A \texttt{JSON} representing the parsed report data.
    \item \textbf{exception:} \texttt{ParseError} if the file format is invalid or parsing fails.
\end{itemize}

\subsubsection{Local Functions}

\noindent \texttt{readLocalReportFile(path: Str)}: 
\begin{itemize}
        \item \textbf{transition:} None (no internal state change.)
        \item \textbf{input:} Path representing location of snippets on user computer.
        \item \textbf{output:} \textit{list [String]} object holding all info from each file.
        \item \textbf{exception:} \texttt{FileError} if file I/O fails or \textit{filePath} is invalid.
\end{itemize}

\noindent \texttt{parseReportContent(rawContent: list[Str])}: Transforms raw file content into a \texttt{JSON}.
\begin{itemize}
        \item \textbf{transition:} None (no internal state change).
        \item \textbf{input:}  \textit{list [String]} object holding all info from each
        file.
        \item \textbf{output:} JSON representing code snippet contents.
        \item \textbf{exception:} None
\end{itemize}

\section{MIS of Threshold Adjustment Module} \label{mThreshold} %MODULE START ####

\subsection{Module}

\texttt{ThresholdModule}

\subsection{Uses}

\begin{itemize}
    \item A back-end or configuration service (to store and retrieve the threshold settings)
    \item A front-end/UI component for user manipulation to achieve their desired threshold
\end{itemize}

\subsection{Syntax}

\subsubsection{Exported Constants}

\begin{itemize}
    \item \texttt{DEFAULT\_THRESHOLD} :  $\mathbb{N}$ \\
    A real number (e.g., 0.75) used if no custom threshold is set.
    \item \texttt{THRESHOLD\_RANGE} : $2-tuple \in \mathbb{R}^ 2 $ \\
    A 2-tuple of real numbers  (e.g., \texttt{(0, 1)}) defining the permissable bounds for a threshold.
\end{itemize}

\subsubsection{Exported Access Programs}

\begin{center}
\begin{tabular}{p{3.5cm} p{3.8cm} p{3cm} p{2.5cm}}
\hline
\textbf{Name} & \textbf{In} & \textbf{Out} & \textbf{Exceptions} \\
\hline
\texttt{getThreshold} & - & \texttt{value: $\mathbb{R}$} & \texttt{ThresholdError} \\
\texttt{setThreshold} & \textit{newVal: $\mathbb{R}$} & \texttt{success: bool} & \texttt{ThresholdError} \\
\texttt{validateThreshold} & \textit{value: $\mathbb{R}$} & \texttt{success: bool} & \texttt{ThresholdError} \\
\hline
\end{tabular}
\end{center}

\subsection{Semantics}

\subsubsection{State Variables}

\begin{itemize}
    \item \texttt{currentThreshold} : $\mathbb{R}$ \\
     Represents the chosen plagiarism detection threshold.
\end{itemize}

\subsubsection{Environment Variables}

\begin{itemize}
    \item \texttt{THRESHOLD\_CONFIG\_ENDPOINT} : \texttt{Str} The network endpoint or file resource where the threshold configuration is stored/persisted.
\end{itemize}

\subsubsection{Assumptions}

\begin{itemize}
    \item \texttt{currentThreshold} is always within \texttt{THRESHOLD\_RANGE}.
    \item Any saved or loaded threshold configurations adhere to the same data format as defined here.
\end{itemize}

\subsubsection{Access Routine Semantics}

\noindent \texttt{getThreshold}():
\begin{itemize}
    \item \textbf{transition:} None (no change to internal state).
    \item \textbf{input:} None
    \item \textbf{output:} Returns the current threshold, i.e., \texttt{currentThreshold}.
    \item \textbf{exception:} \texttt{ThresholdError} if the threshold is undefined or fails to load from persistence.
\end{itemize}

\noindent \texttt{setThreshold}(newVal: $\mathbb{R}$):
\begin{itemize}
    \item \textbf{transition:}
    \begin{itemize}
        \item Uses \texttt{validateThreshold} to check if \textit{newVal} falls within \texttt{THRESHOLD\_RANGE}.
        \item Updates \texttt{currentThreshold} to \textit{newVal} if valid.
        \item Saves the new value to the configuration endpoint or local store.
    \end{itemize}
    \item \textbf{input:} A real number representing user's new desired threshold.
    \item \textbf{output:} \texttt{true} if \textit{newVal} is successfully set; otherwise \texttt{false}.
    \item \textbf{exception:} \texttt{ThresholdError} if \textit{newVal} is out of range or otherwise invalid.
\end{itemize}

\noindent \texttt{validateThreshold}(val: $\mathbb{R}$):
\begin{itemize}
    \item \textbf{transition:} None (does not change internal state).
    \item \textbf{input:} A real number representing user's new desired threshold.
    \item \textbf{output:} \texttt{true} if \textit{value} is in \texttt{THRESHOLD\_RANGE}; otherwise \texttt{false}.
    \item \textbf{exception:} \texttt{ThresholdError} if \textit{value} is malformed (e.g., not a number).
\end{itemize}

\subsubsection{Local Functions}

\begin{itemize}
    \item None
\end{itemize}


\section{MIS of Flagging Module} \label{mFlagging}

\subsection{Module}

\texttt{FlaggingModule}

\subsection{Uses}

\begin{itemize}
    \item Front-end/UI framework for displaying flagged items.
\end{itemize}

\subsection{Syntax}

\subsubsection{Exported Constants}

\begin{itemize}
    \item None
\end{itemize}

\subsubsection{Exported Access Programs}

\begin{center}
\begin{tabular}{p{5cm} p{3.5cm} p{3.5cm} p{2cm}}
\hline
\textbf{Name} & \textbf{In} & \textbf{Out} & \textbf{Exceptions} \\
\hline
\texttt{getFlaggedStatus} & submissionID: Str & status: bool & - \\
\texttt{setFlaggedStatus} & submissionID: Str, flag: bool & - & - \\
\texttt{getAllFlagged} & - & flagged: list[Str] & - \\
\hline
\end{tabular}
\end{center}

\subsection{Semantics}

\subsubsection{State Variables}

\begin{itemize}
    \item \texttt{flaggedSubmissions}: dict[Str, bool] A dictionary of flagged submissions.
\end{itemize}

\subsubsection{Environment Variables}

\begin{itemize}
  \item None
\end{itemize}

\subsubsection{Assumptions}

\begin{itemize}
    \item None
\end{itemize}

\subsubsection{Access Routine Semantics}

\noindent \texttt{getFlaggedStatus(submissionID: str)}:
\begin{itemize}
    \item \textbf{transition:} None (no internal state change.)
    \item \textbf{input:} The unique \texttt{submissionID} for which the flagged status is queried
    \item \textbf{output:} Returns the flagged status (\texttt{true} or \texttt{false}) for the given \texttt{submissionID}
    \item \textbf{exception:} None
\end{itemize}

\noindent \texttt{setFlaggedStatus(submissionID: str, flag: bool)}:
\begin{itemize}
    \item \textbf{transition:}
    \begin{itemize}
        \item Updates the flagged status of the specified \texttt{submissionID} to the provided \texttt{flag} value
    \end{itemize}
     \item \textbf{input:} 
    \begin{itemize}
        \item \texttt{submissionID}: The unique identifier of the submission to update
        \item \texttt{flag}: A boolean value indicating the new flagged status (\texttt{true} or \texttt{false})
    \end{itemize}
    \item \textbf{output:} None.
    \item \textbf{exception:} None
\end{itemize}

\noindent \texttt{getAllFlagged()}:
\begin{itemize}
    \item \textbf{transition:} None (no internal state change.)
    \item \textbf{input:} None
    \item \textbf{output:} Returns a list of all \texttt{submissionID}s that are currently flagged
    \item \textbf{exception:} None
\end{itemize}

\subsubsection{Local Functions}
\begin{itemize}
    \item None
\end{itemize}
% END OF FLAGGING


\section{MIS of Report Results Module} \label{mResults}

\subsection{Module}

\texttt{ResultsModule}

\subsection{Uses}

\begin{itemize}
    \item Frontend for rendering reports visually for users
    \item HTTP client or backend connector (for sending data to the backend)
    \item Code Upload Module to obtain currently parsed code for request
\end{itemize}

\subsection{Syntax}

\subsubsection{Exported Access Program}

\begin{center}
\begin{tabular}{p{5cm} p{4cm} p{3.5cm} p{2cm}}
\hline
\textbf{Name} & \textbf{In} & \textbf{Out} & \textbf{Exceptions} \\
\hline
\texttt{sendDataToBackend} & data : JSON & report: JSON & BackendError \\
\texttt{renderReport} & report: JSON & - & - \\
\hline
\end{tabular}
\end{center}

\subsection{Semantics}

\subsubsection{State Variables}

\begin{itemize}
    \item \texttt{report}: JSON \\
    Info for visuals of plagiarism report.
\end{itemize}

\subsubsection{Environment Variables}

\begin{itemize}
    \item \texttt{BACKEND\_URL}: Str \\ 
    URL endpoint for sending processed data to the backend.
\end{itemize}

\subsubsection{Assumptions}

\begin{itemize}
    \item The backend service is reachable under \texttt{BACKEND\_URL}.
\end{itemize}

\subsubsection{Access Routine Semantics}

\noindent \texttt{sendDataToBackend}(data: JSON):
\begin{itemize}
    \item \textbf{transition:} None (no internal state change.)
    \item \textbf{input:} JSON representing code snippet contents.
    \item \textbf{output:} JSON representing plagiarism analysis on code snippets.
    \item \textbf{exception:} \texttt{BackendError} if backend is unreachable or fails to accept data.
\end{itemize}

\noindent \texttt{renderReport(reports: ReportDataStruct)}:
\begin{itemize}
    \item \textbf{transition:} None (no internal state change) but renders a visual representation of the 
    report in the UI
    \item \textbf{input:} The report data JSON of the report to be rendered
    \item \textbf{output:} None
    \item \textbf{exception:} None
\end{itemize}

\subsubsection{Local Functions}
No local functions are required for this module.

%END OF REPORT RESULTS MODULE

\section{NLP Module}  \label{NLPModule} %Module start here ####

\subsection{Module}

\texttt{NLPModule}

\subsection{Uses}

\begin{itemize}
    \item Abstract ML Model Module
    \item Tokenization Module
    \item AST Module
\end{itemize}

\subsection{Syntax}

\subsubsection{Exported Constants}

\begin{itemize}
    \item None
\end{itemize}

\subsubsection{Exported Access Programs}

\begin{center}
\begin{tabular}{p{5cm} p{3.5cm} p{3.5cm} p{2cm}}
\hline
\textbf{Name} & \textbf{In} & \textbf{Out} & \textbf{Exceptions} \\ 
\hline
\texttt{combinedPredict} & data: JSON & relations: list[dict[Str, $\mathbb{R}$]] & -\\
\hline
\end{tabular}
\end{center}

\subsection{Semantics}

\subsubsection{State Variables}

\begin{itemize}
    \item None
\end{itemize}

\subsubsection{Environment Variables}

\begin{itemize}
  \item None
\end{itemize}

\subsubsection{Assumptions}

\begin{itemize}
    \item The code snippets inputted within any data received comes from one programming language.
\end{itemize}

\subsubsection{Access Routine Semantics}
\noindent \texttt{combinedPredict}(data: JSON):
\begin{itemize}
    \item \textbf{transition:} None (no internal state change.)
    \item \textbf{input:} Data representing inputted code snippets in a JSON format.
    \item \textbf{output:} A variable denoted as relations of type \texttt{list[dict[Str, $\mathbb{R}$]]} that contains an assembly of results from each of the used modules so the results can be combined to get a more balanced perspective of relations between code snippets.
    \item \textbf{exception: None}
\end{itemize}

\subsubsection{Local Functions}
No local functions are required for this module.


%END OF NLP Module

\section{MIS of Abstract Model Module} \label{smMLModel} %Module Start

\subsection{Module}

\texttt{MLModule}

\subsection{Uses}

\begin{itemize}
    \item \texttt{transformers} library
\end{itemize}

\subsection{Syntax}

\subsubsection{Exported Constants}

\begin{itemize}
    \item None.
\end{itemize}

\subsubsection{Exported Access Programs}

\begin{center}
\begin{tabular}{p{5cm} p{3.5cm} p{3.5cm} p{2cm}}
\hline
\textbf{Name} & \textbf{In} & \textbf{Out} & \textbf{Exceptions} \\
\hline
\texttt{train} & data: 2-tuple[list[Str], dict[Str, $\mathbb{R}$]], supervised: bool , timeout: $\mathbb{Z}$  & None & TimeOutException \\
\texttt{predict} & data: list[Str] & result: dict[Str, $\mathbb{R}$] & - \\
\hline
\end{tabular}
\end{center}

\subsection{Semantics}

\subsubsection{State Variables}

\begin{itemize}
    \item weightings: list[$\mathbb{R}$] \\
    Weightings that affect prediction of model
\end{itemize}

\subsubsection{Environment Variables}

\begin{itemize}
  \item None
\end{itemize}

\subsubsection{Assumptions}

\begin{itemize}
    \item The code snippets inputted within any data received comes from one programming language.
\end{itemize}
\subsubsection{Access Routine Semantics}

\noindent \texttt{train(data: 2-tuple[list[Str], dict[Str, $\mathbb{R}$]], supervised: bool, timeout: $\mathbb{R}$)}:
\begin{itemize}
    \item \textbf{transition:} 
    \begin{itemize}
        \item weightings := updated\_weightings -- assign new weightings from batch training to weightings
    \end{itemize}
    \item \textbf{input:} code snippets received for training purposes coupled with predictions which may or may not be empty depending on if learning is supervised or not (\textit{true or false}).
    \item \textbf{output:} None
    \item \textbf{exception:} \texttt{TimeOutException} if training time exceeds time limit allotted. 
\end{itemize}

\noindent \texttt{predict(data: list[Str])}:
\begin{itemize}
    \item \textbf{transition:} None (no internal state change.)
    \item \textbf{input:} Data representing inputted code snippets parsed into string format.
    \item \textbf{output:} Prediction object containing semantic relations between code snippets contained within the DataStruct data.
    \item \textbf{exception:} None
\end{itemize}
% END OF ABSTRACT MODEL MODULE


\section{MIS of Tokenization Modlue} \label{smTokenization}

\subsection{Module}

\texttt{TokenizationModule}

\subsection{Uses}

\begin{itemize}
    \item None
\end{itemize}

\subsection{Syntax}

\subsubsection{Exported Constants}

\begin{itemize}
    \item None
\end{itemize}

\subsubsection{Exported Access Programs}

\begin{center}
\begin{tabular}{p{5cm} p{3.5cm} p{3.5cm} p{2cm}}
\hline
\textbf{Name} & \textbf{In} & \textbf{Out} & \textbf{Exceptions} \\
\hline
\texttt{tokenize} & source: Str & tokens: list[$\mathbb{N}$] & TokenizeError \\
\hline
\end{tabular}
\end{center}

\subsection{Semantics}

\subsubsection{State Variables}

\begin{itemize}
    \item None
\end{itemize}

\subsubsection{Environment Variables}

\begin{itemize}
  \item None
\end{itemize}

\subsubsection{Assumptions}

\begin{itemize}
    \item None
\end{itemize}

\subsubsection{Access Routine Semantics}

\noindent \texttt{tokenize(source: str)}:
\begin{itemize}
    \item \textbf{transition:}
    \begin{itemize}
        \item None
    \end{itemize}
    \item \textbf{input:} A single code snippet.
    \item \textbf{output:} Returns a list of tokens (integers) corresponding to the input source text.
    \item \textbf{exception:} \texttt{TokenizeError} if the source code is syntactically invalid.
\end{itemize}

\subsubsection{Local Functions}

\noindent \texttt{pollOneToken($2-tuple \in \mathbb{N}^2$)}:
\begin{itemize}
    \item \textbf{transition:}
    \begin{itemize}
        \item None
    \end{itemize}
    \item \textbf{output:} Returns a single token read from the given indices of the source string, or None if invalid.
    \item \textbf{exception:} None
\end{itemize}
% END OF TOKENIZATION MODULE

\section{MIS of Abstract Syntax Tree Module} \label{smAST}

\subsection{Module}

\texttt{ASTModule}

\subsection{Uses}

\begin{itemize}
    \item Built in (for python) or external tree parsers
\end{itemize}

\subsection{Syntax}

\subsubsection{Exported Constants}

\begin{itemize}
    \item None
\end{itemize}

\subsubsection{Exported Access Programs}

\begin{center}
\begin{tabular}{p{5cm} p{3.5cm} p{3.5cm} p{2cm}}
\hline
\textbf{Name} & \textbf{In} & \textbf{Out} & \textbf{Exceptions} \\
\hline
\texttt{parse} & rawSource: Str & tree: AST  & invalidSyntaxException \\
\hline
\end{tabular}
\end{center}

\subsection{Semantics}

\subsubsection{State Variables}

\begin{itemize}
    \item None
\end{itemize}

\subsubsection{Environment Variables}

\begin{itemize}
  \item None
\end{itemize}

\subsubsection{Assumptions}

\begin{itemize}
    \item The code snippet provided is syntactically correct or can be parsed with the given rules.
\end{itemize}

\subsubsection{Access Routine Semantics}

\noindent \texttt{parse(rawSource: str)}:
\begin{itemize}
    \item \textbf{transition:} None (no internal state change.)
    \item \textbf{input:} A string representing the input source code
    \item \textbf{output:} Returns the root node of the AST
    \item \textbf{exception:} invalidSyntaxException
\end{itemize}

\subsubsection{Local Functions}
No local functions are required for this module.
% END OF AST MODULE

\section{Similarity Scoring Module} \label{mScoring}

\subsection{Module}

\texttt{SimScoreModule}

\subsection{Uses}

\begin{itemize}
    \item NLP module
\end{itemize}

\subsection{Syntax}

\subsubsection{Exported Constants}

\begin{itemize}
    \item None
\end{itemize}

\subsubsection{Exported Access Programs}

\begin{center}
\begin{tabular}{p{5cm} p{3.5cm} p{3.5cm} p{2cm}}
\hline
\textbf{Name} & \textbf{In} & \textbf{Out} & \textbf{Exceptions} \\
\hline
\texttt{score} & data: JSON & scores: list[2-tuple[str]: $\mathbb{R}$] & - \\ %mathematical notation here, specify n choose 2 scores
\hline
\end{tabular}
\end{center}

\subsection{Semantics}

\subsubsection{State Variables}

\begin{itemize}
    \item None
\end{itemize}

\subsubsection{Environment Variables}

\begin{itemize}
  \item None
\end{itemize}

\subsubsection{Assumptions}

\begin{itemize}
    \item The code snippets inputted within any data received comes from one programming language.
\end{itemize}

\subsubsection{Access Routine Semantics}
\noindent \texttt{score}(data: JSON):
\begin{itemize}
    \item \textbf{transition:} None (no internal state change.)
    \item \textbf{input:} Data containing code snippets inputted from the user in the front end.
    \item \textbf{output:} list with associations between code snippet pairings and a score indiciating
    similarity
    \item \textbf{exception:} None
\end{itemize}

\subsubsection{Local Functions}
No local functions are required for this module.


%END OF SIMILARITY SCORING MODULE

\section{Report Generation Module} \label{mReport}

\subsection{Module}

\texttt{RepGenModule}

\subsection{Uses}

\begin{itemize}
    \item Similarity Scoring Module
\end{itemize}

\subsection{Syntax}

\subsubsection{Exported Constants}

\begin{itemize}
    \item None
\end{itemize}

\subsubsection{Exported Access Programs}

\begin{center}
\begin{tabular}{p{5cm} p{3.5cm} p{3.5cm} p{2cm}}
\hline
\textbf{Name} & \textbf{In} & \textbf{Out} & \textbf{Exceptions} \\
\hline
\texttt{generate} & data: JSON & report: JSON & -\\
\hline
\end{tabular}
\end{center}

\subsection{Semantics}

\subsubsection{State Variables}

\begin{itemize}
    \item None
\end{itemize}

\subsubsection{Environment Variables}

\begin{itemize}
  \item None
\end{itemize}

\subsubsection{Assumptions}

\begin{itemize}
    \item The code snippets inputted within any data received comes from one programming language.
\end{itemize}

\subsubsection{Access Routine Semantics}
\noindent \texttt{generate}(data: JSON):
\begin{itemize}
    \item \textbf{transition:} None (no internal state change.)
    \item \textbf{input:} Data containing code snippets inputted from the user in the front end.
    \item \textbf{output:} JSON object wrapping visuals associated with report and 
    similarity scorings to be received by the front end for displaying analysis to the user.
    \item \textbf{exception: None}
\end{itemize}

\subsubsection{Local Functions}
\noindent \texttt{assembleVisuals}(data: JSON):
\begin{itemize}
    \item \textbf{transition:} None (no internal state change.)
    \item \textbf{input:} Data containing code snippets inputted from the user in the front end.
    \item \textbf{output:} JSON
    \item \textbf{exception:} None
\end{itemize}


%END OF REPORT GENERATION MODULE

\section{MIS of Email Sending Module} \label{mEmail}

\subsection{Module}

\texttt{EmailModule}

\subsection{Uses}

\begin{itemize}
    \item SMTP server or email-sending service (such as SendGrid or Amazon SES)
\end{itemize}

\subsection{Syntax}

\subsubsection{Exported Constants}

\begin{itemize}
    \item None
\end{itemize}

\subsubsection{Exported Access Programs}

\begin{center}
\begin{tabular}{p{5cm} p{3.5cm} p{3.5cm} p{2cm}}
\hline
\textbf{Name} & \textbf{In} & \textbf{Out} & \textbf{Exceptions} \\
\hline
\texttt{sendEmail} & recipient: str, subject: str, body: str, attachments: list[file] & bool & - \\
\hline
\end{tabular}
\end{center}

\subsection{Semantics}

\subsubsection{State Variables}

\begin{itemize}
    \item None
\end{itemize}

\subsubsection{Environment Variables}

\begin{itemize}
    \item \texttt{SMTP\_CONFIG}: Configuration details for connecting to the SMTP server (e.g., host, port, authentication).
\end{itemize}

\subsubsection{Assumptions}

\begin{itemize}
    \item A SMTP server or email-sending service is available and configured correctly
\end{itemize}

\subsubsection{Access Routine Semantics}

\noindent \texttt{sendEmail(recipient: str, subject: str, \\
    body: str, attachments: list[file])}:
\begin{itemize}
    \item \textbf{transition:} None (no internal state change.)
    \item \textbf{input:} 
    \begin{itemize}
        \item \texttt{recipient}: Email address of the primary recipient
        \item \texttt{subject}: Subject of the email
        \item \texttt{body}: The main body content of the email (plain text or HTML)
        \item \texttt{attachments}: A list of file objects to be attached to the email
    \end{itemize}
    \item \textbf{output:} Returns \texttt{true} if the email is sent successfully
    \item \textbf{exception:} Throws \texttt{EmailSendException} if email fails to send.
\end{itemize}


\subsubsection{Local Functions}
No local functions are required for this module.
% END OF EMAIL SENDING MODULE

\newpage

\bibliographystyle {plainnat}
\bibliography {../../../refs/References}

\newpage

\section{Appendix} \label{Appendix}
This section highlights components of the systems that are not modules themselves, but support the other modules in the document.

\section*{Appendix --- Networking Considerations}
\begin{itemize}
    \item Rate limiting is planned to support the Code Upload Module
    \begin{itemize}
        \item This ensures code uploads cannot be sent in excessive amounts, and also prevents DoS attacks.
        \item Rate limiting will also ensure fair processing time for all users.
    \end{itemize}
    \item The report generation module will be hidden behind an API layer in the back end.
    \begin{itemize}
        \item The API will be implemented using a framework such as Django or Flask.
    \end{itemize}
\end{itemize}

\section*{Appendix --- Reflection}

The information in this section will be used to evaluate the team members on the
graduate attribute of Problem Analysis and Design.

\input{../../Reflection.tex}

\begin{enumerate}
  \item \textbf{What went well while writing this deliverable?}   \\
  The process of writing this deliverable was smooth due to the clear structure and guidelines provided. The team collaborated effectively, leveraging each member's strengths. Additionally, the availability of comprehensive documentation and resources facilitated the writing process.

  \item \textbf{What pain points did you experience during this deliverable, and how did you resolve them?}  \\
  One of the main pain points was ensuring consistency across different sections of the document. To resolve this, we conducted regular team meetings to review progress and align on the content. Another challenge was integrating feedback from various team members, which sometimes led to conflicting requirements. We addressed this by prioritizing feedback based on its impact on the project and seeking clarification when necessary.

  \item \textbf{Which of your design decisions stemmed from speaking to your client(s) or a proxy (e.g., your peers, stakeholders, potential users)? For those that were not, why, and where did they come from?}  \\
  None of our existing documents needed to be changed as they were correct upon review.

  \item \textbf{While creating the design doc, what parts of your other documents (e.g., requirements, hazard analysis, etc.), if any, needed to be changed, and why?}  \\
  During the creation of the design document, we identified the need to update the requirements document to reflect changes in the authentication mechanism. Additionally, the hazard analysis document was revised to include potential security risks associated with external service integrations. These changes were necessary to ensure all documents were aligned and accurately represented the current state of the project.

  \item \textbf{What are the limitations of your solution? Put another way, given unlimited resources, what could you do to make the project better?}  \\
  One limitation of our solution is the reliance on external services, which introduces dependencies and potential points of failure. Given unlimited resources, we could develop in-house solutions for critical services to reduce dependency risks. Additionally, we could invest in more robust testing and monitoring tools to enhance the system's reliability and performance. Expanding the team to include specialists in security and performance optimization would also contribute to a more resilient solution.

  \item \textbf{Give a brief overview of other design solutions you considered. What are the benefits and tradeoffs of those other designs compared with the chosen design? From all the potential options, why did you select the documented design?}  \\
  We considered several design alternatives, including using different authentication providers and data storage solutions. For example, we evaluated Firebase Authentication as an alternative to Auth0. While Firebase offers seamless integration with other Firebase services, Auth0 was chosen for its advanced security features and flexibility.
\end{enumerate}



\end{document}