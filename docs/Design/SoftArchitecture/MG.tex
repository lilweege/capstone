\documentclass[12pt, titlepage]{article}

\usepackage{fullpage}
\usepackage[round]{natbib}
\usepackage{multirow}
\usepackage{booktabs}
\usepackage{tabularx}
\usepackage{graphicx}
\graphicspath{ {./UIMockups/} }
\usepackage{float}
\usepackage{hyperref}
\hypersetup{
    colorlinks,
    citecolor=blue,
    filecolor=black,
    linkcolor=red,
    urlcolor=blue
}

\input{../../Comments}
\input{../../Common}

\newcounter{acnum}
\newcommand{\actheacnum}{AC\theacnum}
\newcommand{\acref}[1]{AC\ref{#1}}

\newcounter{ucnum}
\newcommand{\uctheucnum}{UC\theucnum}
\newcommand{\uref}[1]{UC\ref{#1}}

\newcounter{mnum}
\newcommand{\mthemnum}{M\themnum}
\newcommand{\mref}[1]{M\ref{#1}}

\newcounter{smnum}[mnum]
\newcommand{\smthemnum}{\mthemnum.\thesmnum}
\newcommand{\smref}[2]{M\ref{#1}.\ref{#2}}

\begin{document}

\title{Module Guide for \progname{}} 
\author{\authname}
\date{\today}

\maketitle

\pagenumbering{roman}

\section{Revision History}

\begin{tabularx}{\textwidth}{p{3cm}p{2cm}X}
\toprule {\bf Date} & {\bf Version} & {\bf Notes}\\
\midrule
January 7, 2025 & 1.0 & Initial document\\
% Date 2 & 1.1 & Notes\\
\bottomrule
\end{tabularx}

\newpage

\section{Reference Material}

This section records information for easy reference.

\subsection{Abbreviations and Acronyms}

\renewcommand{\arraystretch}{1.2}
\begin{tabular}{l l} 
  \toprule		
  \textbf{symbol} & \textbf{description}\\
  \midrule 
  AC & Anticipated Change\\
  DAG & Directed Acyclic Graph \\
  M & Module \\
  MG & Module Guide \\
  OS & Operating System \\
  R & Requirement\\
  FR & Functional Requirement \\
  SC & Scientific Computing \\
  SRS & Software Requirements Specification\\
  \progname & SyntaxSentinals Code Plagiarism Detector\\
  UC & Unlikely Change \\
  UI & User Interface \\
  NLP & Natural Language Processing \\
  ML & Machine Learning \\
  AST & Abstract Syntax Tree \\
  SSO & Single Sign On \\
  MFA & Multi-Factor Authentication \\
  SDK & Software Development Kit \\
  HTTPS & Hypertext Transfer Protocol Secure \\
  API & Application Programming Interface \\
  JSON & JavaScript Object Notation \\
  

  % \wss{etc.} & \wss{...}\\
  \bottomrule
\end{tabular}\\

\newpage

\tableofcontents

\listoftables

\listoffigures

\newpage

\pagenumbering{arabic}

\section{Introduction}

Decomposing a system into modules is a commonly accepted approach to developing
software.  A module is a work assignment for a programmer or programming
team~ \citep{ParnasEtAl1984}.  We advocate a decomposition
based on the principle of information hiding~ \citep{Parnas1972a}.  This
principle supports design for change, because the ``secrets'' that each module
hides represent likely future changes.  Design for change is valuable in SC,
where modifications are frequent, especially during initial development as the
solution space is explored.  

Our design follows the rules layed out by \citet{ParnasEtAl1984}, as follows:
\begin{itemize}
\item System details that are likely to change independently should be the
  secrets of separate modules.
\item Each data structure is implemented in only one module.
\item Any other program that requires information stored in a module's data
  structures must obtain it by calling access programs belonging to that module.
\end{itemize}

After completing the first stage of the design, the Software Requirements
Specification (SRS), the Module Guide (MG) is developed~\citep{ParnasEtAl1984}. The MG
specifies the modular structure of the system and is intended to allow both
designers and maintainers to easily identify the parts of the software.  The
potential readers of this document are as follows:

\begin{itemize}
\item New project members: This document can be a guide for a new project member
  to easily understand the overall structure and quickly find the
  relevant modules they are searching for.
\item Maintainers: The hierarchical structure of the module guide improves the
  maintainers' understanding when they need to make changes to the system. It is
  important for a maintainer to update the relevant sections of the document
  after changes have been made.
\item Designers: Once the module guide has been written, it can be used to
  check for consistency, feasibility, and flexibility. Designers can verify the
  system in various ways, such as consistency among modules, feasibility of the
  decomposition, and flexibility of the design.
\end{itemize}

The rest of the document is organized as follows. Section
\ref{SecChange} lists the anticipated and unlikely changes of the software
requirements. Section \ref{SecMH} summarizes the module decomposition that
was constructed according to the likely changes. Section \ref{SecConnection}
specifies the connections between the software requirements and the
modules. Section \ref{SecMD} gives a detailed description of the
modules. Section \ref{SecTM} includes two traceability matrices. One checks
the completeness of the design against the requirements provided in the SRS. The
other shows the relation between anticipated changes and the modules. Section
\ref{SecUse} describes the use relation between modules.

\section{Anticipated and Unlikely Changes} \label{SecChange}

This section lists possible changes to the system. According to the likeliness
of the change, the possible changes are classified into two
categories. Anticipated changes are listed in Section \ref{SecAchange}, and
unlikely changes are listed in Section \ref{SecUchange}.

\subsection{Anticipated Changes} \label{SecAchange}

% Anticipated changes are the source of the information that is to be hidden
% inside the modules. Ideally, changing one of the anticipated changes will only
% require changing the one module that hides the associated decision. The approach
% adapted here is called design for
% change.

Anticipated changes are modifications that are likely to occur during the development or maintenance of the system. 
These changes are identified based on the project's goals, stakeholder feedback, and potential future requirements. 
By isolating these changes within specific modules, we ensure that the system remains flexible and maintainable.

The following are the anticipated changes for the SyntaxSentinals Code Plagiarism Detector:

\begin{description}
\item[\refstepcounter{acnum} \actheacnum \label{acInput}:] Changes in the input format of code snippets, such as supporting additional programming languages (e.g., C++, JavaScript) or new file formats.
\item[\refstepcounter{acnum} \actheacnum \label{acNLP}:] Upgrades to the NLP model to improve semantic understanding, such as incorporating newer machine learning techniques or larger training datasets.
\item[\refstepcounter{acnum} \actheacnum \label{acUI}:] Changes in the user interface, such as adding new features (e.g., online learning, language-agnostic support) or improving usability (e.g., better navigation, accessibility features).
\item[\refstepcounter{acnum} \actheacnum \label{acThreshold}:] Adjustments to the similarity threshold for plagiarism detection, allowing users to customize sensitivity levels based on their specific needs.
\end{description}

% \wss{Anticipated changes relate to changes that would be made in requirements,
% design or implementation choices.  They are not related to changes that are made
% at run-time, like the values of parameters.}

\subsection{Unlikely Changes} \label{SecUchange}

The module design should be as general as possible. However, a general system is
more complex. Sometimes this complexity is not necessary. Fixing some design
decisions at the system architecture stage can simplify the software design. If
these decision should later need to be changed, then many parts of the design
will potentially need to be modified. Hence, it is not intended that these
decisions will be changed.

\begin{description}
  \item[\refstepcounter{ucnum} \uctheucnum \label{ucNLP}:] Switching from an NLP-based approach to a non-NLP-based approach.
  \item[\refstepcounter{ucnum} \uctheucnum \label{ucDataRetention}:] Storing user data beyond the immediate task (violating the zero data retention policy).
\end{description}

\section{Module Hierarchy} \label{SecMH}

This section provides an overview of the module design. Modules are summarized
in a hierarchy decomposed by secrets in Table \ref{TblMH}. The modules listed
below, which are leaves in the hierarchy tree, are the modules that will
actually be implemented. \\

\textbf{User Interface Modules}
\begin{description}
  \item [\refstepcounter{mnum} \mthemnum \label{mAuth}:] User Authentication Module
  \item [\refstepcounter{mnum} \mthemnum \label{mCodeUpload}:] Code Upload Module
  \item [\refstepcounter{mnum} \mthemnum \label{mResultsUpload}:] Results Upload Module
  \item [\refstepcounter{mnum} \mthemnum \label{mThreshold}:] Threshold Adjustment Module
  \item [\refstepcounter{mnum} \mthemnum \label{mFlagging}:] Flagging Module
  \item [\refstepcounter{mnum} \mthemnum \label{mResults}:] Report Results Module
\end{description}

\textbf{Backend Modules}
\begin{description}
  \item [\refstepcounter{mnum} \mthemnum \label{mNLP}:] NLP Model Module
  \begin{description}
    \item [\refstepcounter{smnum} \smthemnum \label{smMLModel}:] Abstract ML Model Module
    \item [\refstepcounter{smnum} \smthemnum \label{smTokenization}:] Tokenization Module
    \item [\refstepcounter{smnum} \smthemnum \label{smAST}:] AST Module
  \end{description}
  \item [\refstepcounter{mnum} \mthemnum \label{mScoring}:] Similarity Scoring Module
  \item [\refstepcounter{mnum} \mthemnum \label{mReport}:] Report Generation Module
  \item [\refstepcounter{mnum} \mthemnum \label{mEmail}:] Email Sending Module
\end{description}

\begin{table}[h!]
  \centering
  \begin{tabular}{p{0.35\textwidth} p{0.55\textwidth}}
  \toprule
  \textbf{Level 1} & \textbf{Level 2} \\
  \midrule
  {Hardware-Hiding Module} & ~ \\
  \midrule
  \multirow{2}{*}{Behaviour-Hiding Module} 
  & User Authentication Module \\
  & Code Upload Module \\
  & Results Upload Module \\
  & Report Results Module \\
  & Email Sending Module \\
  & Flagging Module \\
  & Threshold Adjustment Module \\
  \midrule
  \multirow{2}{*}{Software Decision Module} 
  & Report Generation Module \\
  & Similarity Scoring Module \\
  & NLP Model Module \\
  & Abstract ML Model Module \\
  & Tokenization Module \\
  & AST Module \\
 
  \bottomrule
  \end{tabular}
  \caption{Module Hierarchy}
  \label{TblMH}
\end{table}

\section{Connection Between Requirements and Design} \label{SecConnection}

The design of the system is intended to satisfy the requirements developed in
the SRS \citep{SRS}. In this stage, the system is decomposed into modules. This 
section details the connection between functional requirements and modules to 
establish functionality of the design and essentiallity of the modules. The
traceability section below will propose coverage of non-functional requirements 
by the modules. A mapping between functional requirements in the FR section of 
the SRS \citep{SRS} and corresponding modules that provide coverage for them are
also listed in Table~\ref{TblRT}. 
Note: mathematical notation used in the FRs ignores table of abbreviations and acronyms and derives from the SRS section cited above.
\begin{table}[H]
  \centering
  \begin{tabular}{p{0.2\textwidth} p{0.6\textwidth}}
  \toprule
  \textbf{Req.} & \textbf{Modules}\\
  \midrule
  FR-1 & \mref{mCodeUpload} \\
  FR-2 & \mref{mNLP} \mref{mScoring} \mref{mThreshold} \\
  FR-3 & \mref{mResults}\\
  FR-4 & \mref{mFlagging}\\
  FR-5 & \mref{mNLP} \mref{mScoring} \\
  FR-6 & \mref{mReport} \\
  FR-7 & \mref{mAuth} \\
  FR-8 & \mref{mAuth} \\
  FR-9 & \mref{mEmail} \\
  FR-10 & \mref{mResultsUpload} \mref{mResults}\\
  \bottomrule
  \end{tabular}
  \caption{Trace Between Requirements and Modules}
  \label{TblRT}
\end{table}

FR-1 requires all code file inputs given by a user, $u \in U$, to be uploaded
into the system, forming $S$. This necessitates a spot in the UI for users to 
insert code files and get them into circulation of the system for eventual
processing by the model in the back end. These responsibilities will all 
be delegated to the code upload module (\mref{mCodeUpload}).\\

FR-2 requires all snippets to have a pairwise similarity score, Sim($s_i, s_j$), 
and for these scores to be sent against a threshold, $t \in T$, to establish 
plagiarism status. To gain these similarity scores, there must be semantic relation
established between pairs (such as a vector) to calculate scores from. The NLP module 
\mref{mNLP} will be tasked with building the semantic relationship and the Similarity
Scoring module \mref{mScoring} will handle the proceeding score calculation. To translate
similarity scores into plagiarism status, thresholds will need to be obtained from user
preferences and subsequently compared with scores. The Threshold module ({\mref{mThreshold}}) 
will take over these responsibilities necessary for obtaining plagarism status.\\

FR-3 requires a guide document. This will be provided by the results module (\mref{mResults})
in the frontend which will encompass both displaying analysis results to the user as well as 
info on how to interpret them.\\

FR-4 requires allowing the user to flag code snippets for their own tracking purposes
to create $P \subset S$ as well as flagging pairs that are suspected for plagarism. 
This gives scope for a flagging module (\mref{mFlagging}) which will provide the ability 
to mark snippets after analysis for the user to keep special track of and potentially 
filter with.\\

FR-5 requires the collective output of similarity scores, Sim($s_i, s_j$), to exist 
regardless of threshold filtering. This should arise as an output of the scoring module 
(\mref{mScoring}) which will interpret results from the NLP module (\mref{mNLP}).\\

FR-6 requires generation of reports, $R$, using similarity scores, Sim($s_i, s_j$), 
and thresholds that already exists. This necessitates a module which can take a set 
of scores and thresholds that have been obtained from other modules, and make a 
summary report. This will be covered by the Report Generation module (\mref{mReport}).\\

FR-7 requires account creation to add a member, $u$, to the set of users, $U$, necessitating 
the existence of the user authentication module (\mref{mAuth}).\\

FR-8 requires authenticating an account of a user, $u$. This falls into the scope of 
the user authentication module (\mref{mAuth}).\\

FR-9 requires emailing a set of generated reports, $R$, as a zip file, $z$, to clients.
This necessitates a module separate from the report module (\mref{mReport}) to handle the 
business logic of passing around and zipping reports which will be the emailing module
(\mref{mEmail}). \\

FR-10 requires the user to take a generated report they have received via email as a zip file,
$z$, and upload it back to the UI where they can observe the report. This will require a module 
separate from the results module since it will have to re-interprate uploaded reports in a 
zipped state, which will not necessarily have the same detail as the initial result screen
from the result module (\mref{mResults}). Therefore, there will be a Results Upload module
(\mref{mResultsUpload}). \\
  

% \wss{The intention of this section is to document decisions that are made
%   ``between'' the requirements and the design.  To satisfy some requirements,
%   design decisions need to be made.  Rather than make these decisions implicit,
%   they are explicitly recorded here.  For instance, if a program has security
%   requirements, a specific design decision may be made to satisfy those
%   requirements with a password.}

\section{Module Decomposition} \label{SecMD}

Modules are decomposed according to the principle of ``information hiding''
proposed by \citet{ParnasEtAl1984}. The \emph{Secrets} field in a module
decomposition is a brief statement of the design decision hidden by the
module. The \emph{Services} field specifies \emph{what} the module will do
without documenting \emph{how} to do it. For each module, a suggestion for the
implementing software is given under the \emph{Implemented By} title. If the
entry is \emph{OS}, this means that the module is provided by the operating
system or by standard programming language libraries.  \emph{\progname{}} means the
module will be implemented by the \progname{} software.

Only the leaf modules in the hierarchy have to be implemented. If a dash
(\emph{--}) is shown, this means that the module is not a leaf and will not have
to be implemented.

\subsection{Hardware Hiding Modules}

\begin{description}
\item[Secrets:] The data structure and algorithm used to implement the virtual
  hardware.
\item[Services:] Serves as a virtual hardware used by the rest of the
  system. This module provides the interface between the hardware and the
  software. So, the system can use it to display outputs or to accept inputs.
\item[Implemented By:] OS  
\end{description}

\subsection{Behaviour-Hiding Module}

\begin{description}
\item[Secrets:] The contents of the required behaviours.
\item[Services:] Includes programs that provide externally visible behaviour of
  the system as specified in the software requirements specification (SRS)
  documents. This module serves as a communication layer between the
  hardware-hiding module and the software decision module. The programs in this
  module will need to change if there are changes in the SRS.
\item[Implemented By:] --
\end{description}

\subsubsection{User Authentication Module (\mref{mAuth})}

\begin{description}
\item[Secrets:] The authentication and authorization mechanisms.
\item[Services:] Handles user account creation, login, and access control.
\item[Implemented By:] Auth0
\item[Type of Module:] Library
\end{description}

\subsubsection{Code Upload Module (\mref{mCodeUpload})}

\begin{description}
\item[Secrets:] The format and transport of the input data for the model.
\item[Services:] Sets up pathway to receive the input data files and 
tranforms it into the data structure suitable for backend.
\item[Implemented By:] \progname{}
\item[Type of Module:] Abstract Object
\end{description}

\subsubsection{Results Upload Module (\mref{mResultsUpload})}

\begin{description}
\item[Secrets:] The logic to in-take and parse report files as well as display information contained within them.
\item[Services:] Handles visualizing report files for user in front end.
\item[Implemented By:] \progname
\item[Type of Module:] Abstract Object
\end{description}

\subsubsection{Threshold Adjustment Module (\mref{mThreshold})}

\begin{description}
\item[Secrets:] The logic for adjusting plagiarism detection thresholds.
\item[Services:] Allows users to customize the similarity threshold for 
plagiarism detection according to their needs.
\item[Implemented By:] \progname{}
\item[Type of Module:] Abstract Object
\end{description}

\subsubsection{Flagging Module (\mref{mFlagging})}

\begin{description}
\item[Secrets:] The logic for managing and storing the state of user-flagged submissions.
\item[Services:] Allows users to flag specific code snippets or results for further review.
\item[Implemented By:] \progname{}
\item[Type of Module:] Abstract Object
\end{description}

\subsubsection{Report Results Module (\mref{mResults})}

\begin{description}
\item[Secrets:] The logic for requesting and displaying plagiarism
report from the backend.
\item[Services:] Gives spot to initiate plagarism analysis and, upon 
completion, visualizes results to allow user to inspect the findings
of the analysis.
\item[Implemented By:] \progname{}
\item[Type of Module:] Abstract Object
\end{description}

\subsubsection{Email Sending Module (\mref{mEmail})}

\begin{description}
\item[Secrets:] The logic and credentials for sending emails through an email API.
\item[Services:] Handles the sending of emails, including composing the email body and attaching files.
\item[Implemented By:] OS
\item[Type of Module:] Library
\end{description}

\subsection{Software Decision Module}

\begin{description}
\item[Secrets:] The design decision based on mathematical theorems, physical
  facts, or programming considerations. The secrets of this module are
  \emph{not} described in the SRS.
\item[Services:] Includes data structure and algorithms used in the system that
  do not provide direct interaction with the user. 
\item[Implemented By:] --
\end{description}

\subsubsection{NLP Model Module (\mref{mNLP})}

\begin{description}
\item[Secrets:] The NLP-based plagiarism detection algorithms and functions.
\item[Services:] Processes code snippets and generates semantic representations (such as a vector) 
for similarity scoring.
\item[Implemented By:] \progname{}
\item[Type of Module:] Library
\end{description}


\subsubsection{Abstract ML Model Module (\smref{mNLP}{smMLModel})}

\begin{description}
\item[Secrets:] The architecture and configuration of machine learning models.
\item[Services:] Provides a high-level interface for tuning and predicting using various pre-trained machine learning models, agnostic of the specific model to be used.
\item[Implemented By:] Hugging Face
\item[Type of Module:] Abstract Object
\end{description}

\subsubsection{Tokenization Module (\smref{mNLP}{smTokenization})}

\begin{description}
\item[Secrets:] The tokenization algorithms used to break code into smaller, meaningful components for analysis.
\item[Services:] Converts raw code snippets into tokens (e.g., keywords, operators) that can be processed by other modules.
\item[Implemented By:] \progname{}
\item[Type of Module:] Library Component
\end{description}

\subsubsection{AST Module (\smref{mNLP}{smAST})}

\begin{description}
\item[Secrets:] The structure and traversal methods of the Abstract Syntax Tree (AST) used to represent code.
\item[Services:] Transforms source code into an Abstract Syntax Tree for semantic analysis and comparison.
\item[Implemented By:] \progname{}
\item[Type of Module:] Library 
\end{description}

\subsubsection{Similarity Scoring Module (\mref{mScoring})}

\begin{description}
\item[Secrets:] The algorithm for calculating similarity scores.
\item[Services:] Compares semantic representations of code snippets and generates similarity scores.
\item[Implemented By:] \progname{}
\item[Type of Module:] Library
\end{description}

\subsubsection{Report Generation Module (\mref{mReport})}

\begin{description}
\item[Secrets:] The logic for aggregating results of the NLP model and similarity scores.
\item[Services:] Produces a report with both direct and aggregated data from plagarism
analysis.
\item[Implemented By:] \progname{}
\item[Type of Module:] Abstract Object
\end{description}


% \subsection{Hardware Hiding Modules (\mref{mHH})}

% \begin{description}
% \item[Secrets:]The data structure and algorithm used to implement the virtual
%   hardware.
% \item[Services:]Serves as a virtual hardware used by the rest of the
%   system. This module provides the interface between the hardware and the
%   software. So, the system can use it to display outputs or to accept inputs.
% \item[Implemented By:] OS
% \end{description}

% \subsection{Behaviour-Hiding Module}

% \begin{description}
% \item[Secrets:]The contents of the required behaviours.
% \item[Services:]Includes programs that provide externally visible behaviour of
%   the system as specified in the software requirements specification (SRS)
%   documents. This module serves as a communication layer between the
%   hardware-hiding module and the software decision module. The programs in this
%   module will need to change if there are changes in the SRS.
% \item[Implemented By:] --
% \end{description}

% \subsubsection{Input Format Module (\mref{mInput})}

% \begin{description}
% \item[Secrets:]The format and structure of the input data.
% \item[Services:]Converts the input data into the data structure used by the
%   input parameters module.
% \item[Implemented By:] [Your Program Name Here]
% \item[Type of Module:] [Record, Library, Abstract Object, or Abstract Data Type]
%   [Information to include for leaf modules in the decomposition by secrets tree.]
% \end{description}

% \subsubsection{Etc.}

% \subsection{Software Decision Module}

% \begin{description}
% \item[Secrets:] The design decision based on mathematical theorems, physical
%   facts, or programming considerations. The secrets of this module are
%   \emph{not} described in the SRS.
% \item[Services:] Includes data structure and algorithms used in the system that
%   do not provide direct interaction with the user. 
%   % Changes in these modules are more likely to be motivated by a desire to
%   % improve performance than by externally imposed changes.
% \item[Implemented By:] --
% \end{description}

% \subsubsection{Etc.}

\section{Traceability Matrix} \label{SecTM}

This section shows two traceability matrices: between the modules and the
requirements and between the modules and the anticipated changes.

% the table should use mref, the requirements should be named, use something
% like fref
Traceability table between modules and the funcitonal requirements can be found 
at \mref{TblRT}. \\

\begin{table}[H]
  \centering
  \begin{tabular}{p{0.2\textwidth} p{0.6\textwidth}}
  \toprule
  \textbf{Req.} & \textbf{Modules}\\
  \midrule
  UH-L1 & \mref{mResults} \\
  PR-SL1 & \mref{mNLP} \mref{mScoring} \\
  PR-SL2 & \mref{mResults} \\
  PR-PA1 & \mref{mNLP} \mref{mScoring} \\
  PR-PA2 & \mref{mNLP} \mref{mScoring} \\
  PR-RFT1 & \mref{mCodeUpload} \\
  PR-C1 & \mref{mNLP} \mref{mScoring} \\
  PR-L2 & \mref{mNLP} \\
  OE-IAS1 & \mref{mReport} \mref{mNLP} \\
  OE-IAS3 & \mref{mAuth} \\
  MS-M1 & \mref{mNLP} \\
  MS-A2 & \mref{mNLP} \\
  MS-A3 & \mref{mNLP} \\
  SR-A1 & \mref{mAuth} \\
  SR-P1 & \mref{mCodeUpload} \mref{mReport} \mref{mEmail}\\
  SR-P2 & \mref{mCodeUpload} \mref{mEmail}\\
  CR-L1 & \mref{mCodeUpload} \mref{mReport} \mref{mEmail}\\
  CR-L2 & \mref{mCodeUpload} \mref{mReport} \mref{mEmail}\\
  CR-SC4 & \mref{mNLP} \\
  CR-SC5 & \mref{mAuth} \mref{mCodeUpload} \mref{mReport} \\
  \bottomrule
  \end{tabular}
  \caption{Trace Between Relevant Non-Functional Requirements and Modules}
  \label{TblNFRT}
\end{table}

\begin{table}[H]
  \centering
  \begin{tabular}{p{0.2\textwidth} p{0.6\textwidth}}
  \toprule
  \textbf{AC} & \textbf{Modules}\\
  \midrule
  \acref{acInput} & \mref{mCodeUpload} \mref{mNLP}\\
  \acref{acNLP} & \mref{mNLP} \\
  \acref{acUI} & \mref{mResults} \mref{mAuth} \\
  \acref{acThreshold} & \mref{mThreshold} \\
  \bottomrule
  \end{tabular}
  \caption{Trace Between Anticipated Changes and Modules}
  \label{TblACT}
\end{table}
% \begin{table}[H]
% \centering
% \begin{tabular}{p{0.2\textwidth} p{0.6\textwidth}}
% \toprule
% \textbf{Req.} & \textbf{Modules}\\
% \midrule
% R1 & \mref{mHH}, \mref{mInput}, \mref{mParams}, \mref{mControl}\\
% R2 & \mref{mInput}, \mref{mParams}\\
% R3 & \mref{mVerify}\\
% R4 & \mref{mOutput}, \mref{mControl}\\
% R5 & \mref{mOutput}, \mref{mODEs}, \mref{mControl}, \mref{mSeqDS}, \mref{mSolver}, \mref{mPlot}\\
% R6 & \mref{mOutput}, \mref{mODEs}, \mref{mControl}, \mref{mSeqDS}, \mref{mSolver}, \mref{mPlot}\\
% R7 & \mref{mOutput}, \mref{mEnergy}, \mref{mControl}, \mref{mSeqDS}, \mref{mPlot}\\
% R8 & \mref{mOutput}, \mref{mEnergy}, \mref{mControl}, \mref{mSeqDS}, \mref{mPlot}\\
% R9 & \mref{mVerifyOut}\\
% R10 & \mref{mOutput}, \mref{mODEs}, \mref{mControl}\\
% R11 & \mref{mOutput}, \mref{mODEs}, \mref{mEnergy}, \mref{mControl}\\
% \bottomrule
% \end{tabular}
% \caption{Trace Between Requirements and Modules}
% \label{TblRT}
% \end{table}

% \begin{table}[H]
% \centering
% \begin{tabular}{p{0.2\textwidth} p{0.6\textwidth}}
% \toprule
% \textbf{AC} & \textbf{Modules}\\
% \midrule
% \acref{acHardware} & \mref{mHH}\\
% \acref{acInput} & \mref{mInput}\\
% \acref{acParams} & \mref{mParams}\\
% \acref{acVerify} & \mref{mVerify}\\
% \acref{acOutput} & \mref{mOutput}\\
% \acref{acVerifyOut} & \mref{mVerifyOut}\\
% \acref{acODEs} & \mref{mODEs}\\
% \acref{acEnergy} & \mref{mEnergy}\\
% \acref{acControl} & \mref{mControl}\\
% \acref{acSeqDS} & \mref{mSeqDS}\\
% \acref{acSolver} & \mref{mSolver}\\
% \acref{acPlot} & \mref{mPlot}\\
% \bottomrule
% \end{tabular}
% \caption{Trace Between Anticipated Changes and Modules}
% \label{TblACT}
% \end{table}

\section{Use Hierarchy Between Modules} \label{SecUse}

In this section, the uses hierarchy between modules is
provided. \citet{Parnas1978} said of two programs A and B that A {\em uses} B if
correct execution of B may be necessary for A to complete the task described in
its specification. That is, A {\em uses} B if there exist situations in which
the correct functioning of A depends upon the availability of a correct
implementation of B.  Figure \ref{FigUHBE} illustrates the use relation between
the modules. It can be seen that the graph is a directed acyclic graph
(DAG). Each level of the hierarchy offers a testable and usable subset of the
system, and modules in the higher level of the hierarchy are essentially simpler
because they use modules from the lower levels.





% \wss{The uses relation is not a data flow diagram.  In the code there will often
% be an import statement in module A when it directly uses module B.  Module B
% provides the services that module A needs.  The code for module A needs to be
% able to see these services (hence the import statement).  Since the uses
% relation is transitive, there is a use relation without an import, but the
% arrows in the diagram typically correspond to the presence of import statement.}

% \wss{If module A uses module B, the arrow is directed from A to B.}

\begin{figure}[H]
\centering
\includegraphics[width=0.5\textwidth]{../Assets/BackEndHierarchy.png}
\caption{Use hierarchy among back end modules}
\label{FigUHBE}
\end{figure}



\begin{figure}[H]
  \centering
  \includegraphics[width=0.5\textwidth]{../Assets/FrontEndHierarchy.png}
  \caption{Use hierarchy among front end modules}
  \label{FigUHFE}
  \end{figure}

%\section*{References}
\newpage
\section{User Interfaces}
The below are mockups of the user interface for the software and are subject to change based on user feedback and design decisions.

\begin{figure}[H]
  \centering
  \includegraphics[width=\textwidth]{Landing.png}
  \caption{Landing Page}
  \label{fig:landing}
\end{figure}

\begin{figure}[H]
  \centering
  \includegraphics[width=\textwidth]{home.png}
  \caption{Home Page}
  \label{fig:home}
\end{figure}

\begin{figure}[H]
  \centering
  \includegraphics[width=\textwidth]{results.png}
  \caption{Results Page}
  \label{fig:results}
\end{figure}

\begin{figure}[H]
  \centering
  \includegraphics[width=\textwidth]{settings.png}
  \caption{settings Page}
  \label{fig:results}
\end{figure}

\section{Design of Communication Protocols}

This section outlines the external services that will be used by the software and their respective purposes.

\subsection{External Services Overview}
The system will integrate with Auth0 to enhance functionality and provide a seamless user experience.

\subsection{Authentication Service: Auth0}
Auth0 will be used for user authentication and authorization. It provides secure login, single sign-on (SSO), and multi-factor authentication (MFA) capabilities. Auth0 will handle user credentials and ensure that only authorized users can access the system.

\subsection{Integration and Security}
Auth0 will be integrated using its Javascript SDK. Secure communication channels (e.g., HTTPS) will be used to protect data in transit. Additionally, proper authentication and authorization mechanisms will be implemented to ensure that only authorized components can interact with Auth0.

\section{Timeline}

% \wss{Schedule of tasks and who is responsible}

\subsection{Module Implementation Timeline (Starting Jan 17)}

\subsubsection{User Interface (UI) Modules}
\begin{itemize}
    \item \textbf{Jan 17 - Jan 20: User Authentication Module (\ref{mAuth})}
    \begin{itemize}
        \item Develop UI for user registration and login.
        \item Integrate with backend authentication services.
        \item \textbf{Responsibility:} Lucas.
    \end{itemize}
    
    \item \textbf{Jan 21 - Jan 25: Code Upload Module (\ref{mCodeUpload})}
    \begin{itemize}
        \item Implement the file upload interface.
        \item Validate and display upload status to the user.
        \item \textbf{Responsibility:} Mohsin.
    \end{itemize}
    
    \item \textbf{Jan 26 - Jan 30: Results Upload Module (\ref{mResultsUpload})}
    \begin{itemize}
        \item Create UI for uploading processed result files.
        \item Display confirmation and errors for invalid uploads.
        \item To finalize this module it requires the completion of \ref{mEmail}
        \item \textbf{Responsibility:} Lucas.
    \end{itemize}
    
    \item \textbf{Jan 31 - Feb 4: Threshold Adjustment Module (\ref{mThreshold})}
    \begin{itemize}
        \item Develop an interface for adjusting similarity thresholds.
        \item Provide sliders or input boxes for customization.
        \item \textbf{Responsibility:} Mohsin.
    \end{itemize}
    
    \item \textbf{Feb 5 - Feb 8: Flagging Module (\ref{mFlagging})}
    \begin{itemize}
        \item Add functionality for users to flag suspicious results.
        \item Ensure flagged items are visually distinct in the UI.
        \item \textbf{Responsibility:} Mohsin.
    \end{itemize}
    
    \item \textbf{Feb 9 - Feb 14: Report Results Module (\ref{mResults})}
    \begin{itemize}
        \item Implement a results display interface with options for sorting and filtering.
        \item Add the ability to download or email reports directly from the UI.
        \item \textbf{Responsibility:} Lucas and Mohsin.
    \end{itemize}
\end{itemize}

\subsubsection{Backend Modules}
\begin{itemize}
    \item \textbf{Jan 17 - Jan 23: NLP Model Module (\ref{mNLP})}
    \begin{itemize}
        \item Finalize and integrate the trained plagiarism detection model.
        \item \textbf{Responsibility:} Dennis, Luigi, and Julian.
    \end{itemize}

    \item \textbf{Jan 17 - Jan 29: Similarity Scoring Module (\ref{mScoring})}
    \begin{itemize}
        \item Process NLP model outputs for similarity scores.
        \item Optimize backend logic for performance.
        \item \textbf{Responsibility:} Dennis, Luigi, and Julian.
    \end{itemize}

    \item \textbf{Jan 30 - Feb 4: Report Generation Module (\ref{mReport})}
    \begin{itemize}
        \item Generate reports based on processed similarity data.
        \item Ensure compatibility with frontend requirements.
        \item \textbf{Responsibility:} Entire team.
    \end{itemize}

    \item \textbf{Feb 5 - Feb 8: Email Sending Module (\ref{mEmail})}
    \begin{itemize}
        \item Develop functionality for sending results via email.
        \item Implement error handling for failed deliveries.
        \item \textbf{Responsibility:} Dennis, Luigi, and Julian.
    \end{itemize}
\end{itemize}

\vspace{0.5cm}


% \wss{You can point to GitHub if this information is included there}

\bibliographystyle {plainnat}
\bibliography{../../../refs/References}

\newpage{}

\end{document}